\documentclass[a4paper, 11pt]{article}
\usepackage{lipsum} %This package just generates Lorem Ipsum filler text. 
\usepackage{fullpage} % changes the margin
\usepackage{mathpazo}
\usepackage{multicol}
\usepackage{graphicx}
\usepackage{enumerate}
\usepackage{amsmath,amsfonts,amsthm} % Math packages
\usepackage{listings}

\begin{document}
%Header-Make sure you update this information!!!!
\noindent
\large\textbf{Homework 5} \hfill \textbf{Hongyu Yan (516030910595)} \\
\normalsize {\bf CS 259 Numerical Methods for Data Science} \hfill ACM Class, Zhiyuan College, SJTU\\
Prof.~{\bf David Bindel} \hfill Due Date: June 27, 2018\\
TA.~{\bf Yurong You, Xinran Zhu} \hfill Submit Date: \today

\section*{Problem 1}
As $A$ is symmetric, the difference between $\Lambda$ and $\Sigma$ is that
the values in the diagonal of $\Sigma$ are non-negative and in descending order,
while those in $\Lambda$ are not necessarily positive and unordered.

To compute the SVD, we can make it by change the sign of negative entries of $\Lambda$
and sort them in descending order. 

So the alogrithm can be described as:
\begin{enumerate}
\item Generate proper matrix $L$ s.t. $L\Lambda$ is non-negative. In other words, if $\Lambda_{ii} \leq 0$, then $L_{ii} = -1$, otherwise $L_{ii} = 1$.
\item Generate proper elementary matrix $T$ s.t. $T(L\Lambda)T$ is the matrix that
the values in the diagonal are in descending order.
\item So that $\Sigma = TL\Lambda T$, and we have $A = Q\Lambda Q^T = Q(TL)^{-1}(TL\Lambda T)T^{-1}Q = Q(TL)^{-1}\Sigma T^{-1}Q$. So $U = Q(TL)^{-1}$, $V=(T^{-1}Q)^T$.
\end{enumerate}
The time complexity of this algorithm is $O(n^2)$, because generating the sorted matrix needs $O(n^2)$.

By the way, I think it's also ok by sorting values in $O(nlogn)$ then generating the matrix $T$ in $O(n)$.
\section*{Problem 2}

Suppose there are n points $x_1, x_2, \cdots, x_n$, so $A \in \mathbb{R}^{n \times n}$. 
$ X = 
\begin{bmatrix}
x_1^T \\
x_2^T \\
\vdots \\
x_n^T
\end{bmatrix}
$, $X \in \mathbb{R}^{n \times d}$

The purpose is to prove that $B = XX^T$, and we have one solution for $X$ as 
$X = \begin{bmatrix}
B_{11} \\
B_{12} 
\end{bmatrix} R_{11}^{-1}$.

First, from the definition of B, we got:

\begin{align*}
B & = -\frac{1}{2}JAJ \\
  & = -\frac{1}{2}(I-\frac{ee^T}{n})A(I-\frac{ee^T}{n}) \\
  & = -\frac{1}{2}(A-A\frac{ee^T}{n} - \frac{ee^T}{n}A + \frac{ee^T}{n}A\frac{ee^T}{n})
\end{align*}

Specifically, we have:
$$
B_{ij} = -\frac{1}{2}(A_{ij} - \frac{1}{n}\sum_{k=1}^n A_{ik}-\frac{1}{n}\sum_{l=1}^n A_{lj}+\frac{1}{n^2} \sum_{k=1}^n \sum_{l=1}^n A_{lk})
$$

Because $A_{ij} = \| x_i - x_j \| ^2 = \| x_i \|^2-2<x_i,x_j>+\|x_j\|^2$, substitute it into previous equation, we get:
$$
B_{ij}=-\frac{1}{2}(A_{ij}-\|x_i\|^2-\|x_j\|^2)=<x_i,x_j>
$$
So $B=XX^T$ and B is positive semi-definite.

Second, we will prove that $X = \begin{bmatrix}
B_{11} \\
B_{12} 
\end{bmatrix} R_{11}^{-1}$ is a solution.

Suppose there are d "landmark" points, then we can split $X$ into $\begin{bmatrix}
X_1^T \\ X_2^T
\end{bmatrix}$, where $X_1$ contains the d "landmark" points, and $X_2$ contains the other $n-k$ points. So $$ B = XX^T = \begin{bmatrix}
X_1^T \\ X_2^T
\end{bmatrix} \begin{bmatrix}
X_1 & X_2
\end{bmatrix}= \begin{bmatrix}
X_1^TX_1& X_1^TX_2 \\
X_2^TX_1& X_2^TX_2 
\end{bmatrix}= \begin{bmatrix}
B_{11} & B_{12} \\
B_{21} & B_{22} 
\end{bmatrix} $$
 $$B_{12} = B_{21}^T $$

Since $B_{11} = R_{11}^TR_{11} $, so
\begin{align*}
B_{11} &= R_{11}^TR_{11} \\
B_{12} &= R_{11}^T(B_{21}R_{11}^{-1})^T \\
B_{21} &= B_{21}R_{11}^{-1}R_{11} \\
B_{22} &= X_2^T X_2 \\
	   &= X_2^T X_1 (X_1^T X_1)^{-1} X_1^T X_2 \\
	   &= X_2^T X_1(B_{11})^{-1} X_1^T X_2 \\
	   &= B_{21}R_{11}^{-1}(R_{11}^T)^{-1}B_{21}^T \\
	   &= (B_{21}R_{11}^{-1})(B_{21}R_{11}^{-1})^T
\end{align*}
so that:
$$
B=\begin{bmatrix}
R_{11}^T R_{11} & R_{11}^T(B_{21}R_{11}^{-1})^T \\
B_{21}R_{11}^{-1}R_{11} & (B_{21}R_{11}^{-1})(B_{21}R_{11}^{-1})^T 
\end{bmatrix} = \begin{bmatrix}
B_{11} \\ B_{21}
\end{bmatrix} R_{11}^{-1} ( \begin{bmatrix}
B_{11} \\ B_{21}
\end{bmatrix} R_{11}^{-1})^T
$$

\section*{Problem 3}

\begin{align*}
\| A-LMR \|_F^2 &= tr((A-LMR)^T(A-LMR)) \\
			    &= tr((A^T-R^TM^TL^T)(A-LMR)) \\
			    &= tr(A^TA-R^TM^TL^TA-A^TLMR+R^TM^TL^TLMR) \\
			    &= tr(A^TA)-2tr(R^TM^TL^TA)+tr(R^TM^TL^TLMR)
\end{align*}
We get the minimum when the derivative of $\| A-LMR\|_{F}^2 $ equals 0.
Because
\begin{align*}
 \frac{\partial tr(A^TA)}{\partial M} &= 0 \\
 \frac{\partial tr(R^TM^TL^TA)}{\partial M} &= \frac{\partial tr(M^TL^TAR^T)}{\partial M} = L^TAR^T \\
 \frac{\partial tr(R^TM^TL^TLMR)}{\partial M} &= 2L^TLMRR^T
\end{align*}
so that
$$
\frac{\partial \| A-LMR \|_F^2}{\partial M} = -2L^TAR^T + 2L^TLMRR^T=0
$$
which means
\begin{align*}
L^TAR^T &= L^TLMRR^T \\
M &= (LTL)^{-1}L^T A R^T(RR^T)^{-1} = L^\dagger A R^\dagger
\end{align*}
\end{document}
